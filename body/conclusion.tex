\chapter[结论与展望]{结论与展望{\song\xiaosi \textcolor{red}{(黑体小二号加粗,段后1行,居中)}}}

结论是对整个论文主要成果的总结,在\textbf{结论中应明确指出本研究内容的创造性成果或创新点理论(含新见解、新观点)},
对其应用前景和社会、经济价值等加以预测和评价,并指出今后进一步在本研究方向进行研究工作的展望与设想。应准确、完整、明确和精练。

\section{参考文献示例}
\v{S}mejkal等人用群论描述交错磁性\cite{altermagnetic1,altermagnetic2},Zhang等人提出$\cdots\cdots$,详见文献\inlinecite{CrXO}和\inlinecite{OsO2}。

{\color{red}

    参考文献说明:
    
    (内容用宋体、Times New Roman,小四号,1.25倍行距)

[1]	[期刊文章] 作者(三人以内的全列出,超出3人的用“等”字代替). 文题[J]. 刊名,年,卷(期):起始页码-终止页码.

[2]	[专著] 作者. 书名[M]. 出版地:出版社,出版年:起始页码-终止页码

[3]	[译著] 作者. 书名[M]. 译者. 出版地:出版者,出版年.

[4]	[会议论文集] 编者. 文集[C]. 出版地:出版者,出版年. 起始页码-终止页码.

[5]	[析出文献] 作者. 析出文献题名[C]∥原文献主要责任者. 原文献题名:其他题名信息(如副标题). 出版地:出版者,出版年:析出文献起始页码-终止页码.

[6]	[学位论文] 作者. 文题[D]. 所在城市:保存单位,年份.

[7]	[专利] 申请者. 专利名: 国名,专利号[P].发布日期. 

[8]	[技术标准] 标准编号, 标准名称[S].

[9]	[技术报告] 作者. 文题[R]. 报告代码及编号,地名:责任单位,年份. 

[10]	[在线文献] 作者. 文题[EB/OL].(公告日期)[引用日期]. http://…, 日期.

[11]	[外文参考文献](英文参考文献中,作者姓名均姓前名后,姓写全称,名字缩写;论文题名首字母大写,其余小写,期刊名称每个单词首字母大写(介词和连词小写);杂志名要求写全称。)

[12]	已公开发表的文献不得用预印版链接格式引用
}
