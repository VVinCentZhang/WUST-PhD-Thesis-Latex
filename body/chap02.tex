
%%% Local Variables:
%%% mode: latex
%%% TeX-master: t
%%% End:

\chapter[公式撰写规范]{公式撰写规范{\song\xiaosi \textcolor{red}{(黑体小二号加粗,段后1行,居中)}}}

\section[二级标题]{二级标题{\song\xiaosi \textcolor{red}{(黑体小三号字加粗,前后段间距0.5行,不空格)}}}
正文段落格式为:宋体/Times New Roman 小四号字,1.25倍行距,首行缩进2字符

\subsection[三级标题]{三级标题{\song\xiaosi \textcolor{red}{(黑体四号字加粗,前后段间距0.25行,不空格)}}}
公式使用公式编辑器输入,公式按章编号,用括号写在右边行末,公式与正文之间要有一行的间距。如:

\begin{equation}
  \phi=\frac{D_{\rm p}^2}{150}\frac{\varPsi^3}{(1-\varPsi)^2}
\end{equation}

公式与正文之间要有一行的间距

\begin{equation}
  C_2=\frac{3.5}{D_{\rm p}}\frac{(1-\varPsi)}{\varPsi^3}
\end{equation}

\begin{align*}
  \text{式中,}D_{\rm p}&\text{—— 多孔质材料的平均粒子直径(m);}\\
  \varPsi&\text{—— 孔隙度(孔隙体积占总体积的百分比)。}
\end{align*}

\textcolor{red}{(要求公式当中字体为Times New Roman,变量为斜体,常量为正体)}
